\section*{The Machine Learning Landscape}
\subsection*{What is Machine Learning?}
Machine Learning is the science (and art) of programming computers so they can
learn from data.\\
Here is a slightly more general definition:
[Machine Learning is the] field of study that gives computers the ability to learn
without being explicitly programmed.\\
—Arthur Samuel, 1959\\
And a more engineering-oriented one:
A computer program is said to learn from experience E with respect to some task T
and some performance measure P, if its performance on T, as measured by P,
improves with experience E.\\
—Tom Mitchell, 1997

\subsection*{Why use Machine Learning?}
To summarize, Machine Learning is great for:\\
• Problems for which existing solutions require a lot of fine-tuning or long lists of
rules: one Machine Learning algorithm can often simplify code and perform better than the traditional approach.\\
• Complex problems for which using a traditional approach yields no good solution: the best Machine Learning techniques can perhaps find a solution.\\
• Fluctuating environments: a Machine Learning system can adapt to new data.\\
• Getting insights about complex problems and large amounts of data.\\

\subsection*{Types of Machine Learning Systems}
There are so many different types of Machine Learning systems that it is useful to
classify them in broad categories, based on the following criteria:\\
• Whether or not they are trained with human supervision (supervised, unsuper‐
vised, semisupervised, and Reinforcement Learning)\\
• Whether or not they can learn incrementally on the fly (online versus batch
learning)\\
• Whether they work by simply comparing new data points to known data points,
or instead by detecting patterns in the training data and building a predictive
model, much like scientists do (instance-based versus model-based learning)\\

\subsection*{Supervised/Unsupervised Learning}
In \textbf{\textit{supervised learning}}, the training set you feed to the algorithm includes the desired
solutions, called labels\\
Here are some of the most important supervised learning algorithms (covered in this
book):\\
• k-Nearest Neighbors\\
• Linear Regression\\
• Logistic Regression\\
• Support Vector Machines (SVMs)\\
• Decision Trees and Random Forests\\
• Neural networks\\

In \textbf{\textit{unsupervised learning}} , the training data is unlabeled. The system tries to learn without a teacher.
Here are some of the most important unsupervised learning algorithms :\\
• Clustering\\
— K-Means\\
— DBSCAN\\
— Hierarchical Cluster Analysis (HCA)\\
• Anomaly detection and novelty detection\\
— One-class SVM\\
— Isolation Forest\\
• Visualization and dimensionality reduction\\
— Principal Component Analysis (PCA)\\
— Kernel PCA\\
— Locally Linear Embedding (LLE)\\
— t-Distributed Stochastic Neighbor Embedding (t-SNE)\\
• Association rule learning\\
— Apriori\\
— Eclat\\
Since labeling data is usually time-consuming and costly, you will often have plenty of
unlabeled instances, and few labeled instances. Some algorithms can deal with data
that’s partially labeled. This is called \textbf{\textit{semisupervised learning}}\\

\textbf{\textit{Reinforcement Learning}} is a very different beast. The learning system, called an agent
in this context, can observe the environment, select and perform actions, and get
rewards in return or penalties.\\
It must then learn by itself what is the best strategy, called a policy, to get
the most reward over time. A policy defines what action the agent should choose
when it is in a given situation.\\

\subsection*{Batch and Online Learning}
In \textbf{\textit{batch learning}}, the system is incapable of learning incrementally: it must be trained
using all the available data. This will generally take a lot of time and computing
resources, so it is typically done offline. First the system is trained, and then it is
launched into production and runs without learning anymore; it just applies what it
has learned. This is called \textit{offline learning}.\\

In \textbf{\textit{online learning}}, you train the system incrementally by feeding it data instances
sequentially, either individually or in small groups called mini-batches. Each learning
step is fast and cheap, so the system can learn about new data on the fly, as it arrives.
One important parameter of online learning systems is how fast they should adapt to
changing data: this is called the \textit{learning rate}.\\

\subsection*{Instance-Based Versus Model-Based Learning}
One more way to categorize Machine Learning systems is by how they \textit{generalize}.
\textbf{\textit{instance-based learning}}: the system learns the examples by heart, then
generalizes to new cases by using a similarity measure to compare them to the
learned examples (or a subset of them).\\
Another way to generalize from a set of examples is to build a model of these exam‐
ples and then use that model to make predictions. This is called \textbf{\textit{model-based learning}}


